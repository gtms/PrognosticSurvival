\section{Software Environment}
All analysis carried in this study were performed in the \textsf{R} software
environment for statistical computing,\cite{r-manual} version 3.1.2, on a
\textsf{Unix} system.  Refer to the \textsf{README.md} file for instructions on
how to setup an environment for reproduction of this analysis.

\section{Datasets}
A total of 114 public data-sets of cancer gene expression profiles spanning 22
types of cancer were downloaded from public repositories, manually curated and
pre-processed for downstream analysis.  Sources for data-sets include the
\href{http://www.ncbi.nlm.nih.gov/geo/}{Gene Expression
  Omnibus}\cite{edgar_gene_2002},
\href{https://insilicodb.com/}{InSilicoDB}\cite{taminau_insilicodb:_2011} and
the \href{http://cancergenome.nih.gov/}{TCGA Research Network site}.  Individual
data-sets are described in the file \textsf{studies.csv} (located under
\textsf{data/csv/}) and are packed as \textsf{Rda} objects (located under
\textsf{data/rda/}).  Data-sets can be loaded into an \textsf{R} environment
with the function \texttt{loadDset} (refer to the README.md file for details on
how to setup an \textsf{R} environment for reproduction of the analysis).  Upon
loading a particular study into your \textsf{R} session, you will find a list
named \textsf{dset} in the global environment.

Clinical data associated with individual data-sets was manually curated and
formatted into a \textsf{data frame} object, labeled \texttt{dset\$pheno}.
Available clinical outcomes for every patient can be found in this \textsf{data
  frame}.  These may include: overall survival (\textsc{os}, reported event is
death); disease-specific survival (\textsc{dss}, reported event is death from
disease); distant-metastasis free survival (\textsc{dmfs}, reported event is
occurrence of distant metastasis); and disease-free survival (\textsc{dfs},
reported event is relapse from disease).  Studies sourced from InSilicoDB were
\textsc{frma} normalized\cite{mccall_frozen_2010}; the remaining studies were
used with their original normalizations.  Gene annotation for each study was
updated using Bioconductor and can be retrieved as \texttt{dset\$ga}
\textsf{list} objects.  Gene symbols were retained as the feature descriptors
for all data-sets.  The expression matrix of each study went through a round of
feature collapsing using a maxSum routine (function
\texttt{collaspseRows})\cite{miller_strategies_2011}, and was median polished
(function \texttt{medianPolish}) as suggested in
\cite{ramaswamy_molecular_2003,venet_most_2011}.  Expression matrices were then
stored in the \textsf{gct} format
(\href{http://www.broadinstitute.org/cancer/software/genepattern/gp\_guides/file-formats/sections/gct}{http://www.broadinstitute.org/})
and can be accessed as \texttt{dset\$gct} objects.

\section{\mbox{MSigDB C2} signatures}
In order to quantify the prognostic signals of each cancer data-set, we
evaluated the association with clinical outcome of the 4722 curated gene sets
from the \href{http://www.broadinstitute.org/gsea/msigdb/index.jsp}{Molecular
  Signatures Database} (\mbox{MSigDB C2}).  \mbox{MSigDB C2} signatures are
manually curated from the literature on gene expression and also include gene
sets from curated pathways databases such as KEGG.

Version 4.0 of the MSigDB curated gene sets (updated on May 31, 2013) was
downloaded as a \textsf{gmt} file (located under \textsf{data/sigs/}) and parsed
with a preprocessing script (\textsf{parse-msigdb-sigs.R}).  This script outputs
two lists of signatures (saved in \textsf{cache/mSigDBc2.RData}): one containing
all original 4722 signatures with gene symbols as identifiers; and another
containing 4722 signatures of the same size as the originals, but composed of
random samplings from the total pool of genes in the collection.

\section{Estimating outcome association}

Estimating association between a genomic marker and clinical outcome requires
the formulation of an outcome predictor from multi-gene values of expression.
Traditionally, this is achieved by a function that stratifies the cohort in good
and bad prognosis groups according to the expression values of the marker.
Venet et al.\cite{venet_most_2011} used a procedure that consisted in bisecting
each cohort along the first principal component of the marker's gene expression
signature.

For simplification purposes, we chose to use the first principal component
loadings of each expression signature as a non-discretized predictor instead.
Each continuous predictor was used to estimate the hazard ratio (\textsc{hr})
and the related log-rank \emph{p}-values of association with outcome using the
standard Cox procedure implemented in the package \textsf{survival} (function
\texttt{logRankTest}).

\subsection{Multi-gene signatures}
For each data-set, we tested the association of all \mbox{MSigDB C2} signature
with all clinical outcomes available using the procedure described above.  Only
signatures with at least five genes profiled in the each data-set microarray
were considered.  For each experimental setting, we then computed the fraction
of \mbox{MSigDB C2} signatures significantly associated to outcome at a
\emph{p-}value of 0.05.  To control for potential biases towards cancer-related
signatures, the same procedure was repeated with randomized \mbox{MSigDB C2}
signatures.  The results of this computation can be found cached as \textsf{R}
objects (saved as \textsf{cache/fracSignif.dfr.RData}).

\subsection{Individual genes}
For each data-set, we tested the association of each individual gene profiled
with all clinical outcomes available using the procedure described above.
Continuous values of gene expression were used as predictors in the tested Cox
models.  For each experimental setting, we then computed the fraction of genes
significantly associated to outcome at a \emph{p-}value of 0.05.  The results of
this computation can be found cached as \textsf{R} objects (saved as
\textsf{cache/fracGenesSignif.dfr.RData}).

\section{Down-sampling experiments on the METABRIC data-set}
The ``discovery'' and ``validation'' cohorts of the METABRIC data-set were
merged, yielding a total of 1972 samples with patient follow-up (object
\texttt{metabric.dset}).  The following experiments were run (prognostic
fraction refers to the fraction of genes in the merged data-set that are
significantly associated with disease-specific survival, at at \emph{p}--value
of 0.05; \textsf{R} code underlying Fig. 2 is available in
\textsf{lib/metabric-analysis.R} and \textsf{src/metabric.R}):
\begin{itemize}
\item \emph{Sampling effect.}  The gene-wise prognostic fraction was computed
  for 100 randomly selected subsets of 500 samples.
\item \emph{Effect of sample size.}  For each \emph{N} $\in$ \{100, 200, 300,
  500, 750, 1000, 1250, 1500, 1750\}, the prognostic fraction was computed for
  100 subsets of \emph{N} samples selected at random.
\item \emph{Effect of follow-up time.}  For each \emph{t} $\in$ \{25, 50, 75,
  100, 125, 150, 175, 200, $max_{cohort}$(follow-up time)\} months, all
  follow-up times $>$ \emph{t} were set to \emph{t} and the respective events
  were set to 0 (i.e., no event).  The prognostic fraction was then computed for
  100 subsets of 500 patients.
\item \emph{Effect of ER status.}  For each fraction \emph{E} of ER positive
  patients (as defined by the vector
  \texttt{metabric.dset\$pheno\$ER\_IHC\_status}) $\in$ \{0, 0.1, 0.2, \ldots{},
  0.9, 1\}, we randomly selected 300 $\times$ \emph{E} ER+ samples and \mbox{300
    $\times$ (1 -- \emph{E})} ER-- samples and computed the respective
  prognostic fraction.  This procedure was repeated 100 times for each value of
  \emph{E}.
\item \emph{Effect of nodal status.}  For each fraction \emph{L} of node
  positive patients (defined as
  \texttt{metabric.dset\$pheno\$lymph\_node\_positive} $>$ 0) $\in$ \{0, 0.1,
  0.2, \ldots{}, 0.9, 1\}, we randomly selected 500 $\times$ \emph{L} node
  positive and 500 $\times$ (1 -- \emph{L}) node negative samples and computed
  the respective prognostic fraction.  This procedure was repeated 100 times for
  each value of \emph{L}.
\item \emph{Effect of cellularity.}  For each cellularity class (defined by the
  three levels of the factor \texttt{metabric.dset\$pheno\$cellularity}) we
  computed the prognostic fraction in 100 subsets of 175 samples each.
\end{itemize}

\section{Re-analysis of GSE9893}
The \textsf{R} code underlying the re-analysis of the data-set
\textsf{GSE9893-breast} can be found in \textsf{src/reanalysis-GSE9893.R}.  Both
the code and the output figures can be found in the pdf document
\textsf{reports/GSE9893-reanalysis/supplementary-reanalysis-GSE9893.pdf}.

The data-set-wise signal-to-noise metrics were computed with the function
\texttt{qualStudy} of the \textsf{R} package \textsf{SNAGEE}
\cite{venet_measure_2012}.  In short, \textsf{SNAGEE}'s signal-to-noise metrics
measures the similarity of the gene-gene correlation matrix of a dataset with a
universal correlation matrix computed from a very large number of studies.
Data-sets with poor gene annotation or with strong batch effects will depart
from this universal matrix and score poorly.  Scan dates were extracted from the
``DateTime'' field available in GenePix scanner \texttt{*.gpr} files available
in the archive \textsf{GSE9693\_RAW.tar} provided in NCBI's GEO.  The survival
analysis of scan dates was performed with the \texttt{coxph} function of
\textsf{R}'s \textsf{survival} package.  Principal components analysis was
performed with \textsf{R}'s \texttt{prcomp} function.  Raw expression data were
obtained from the \texttt{*.gpr} files and quantile-normalized with the function
\texttt{normalizeBetweenArrays} from \textsf{R}'s \textsf{limma} package.
Quantile normalization normalizes expression measure distributions across all
arrays.  All arrays were processed as a single batch.

%%% Local Variables:
%%% mode: latex
%%% TeX-master: t
%%% End:
